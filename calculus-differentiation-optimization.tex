\documentclass[20150903-160354-rs2.2-MarksMathNotebook.tex]{subfiles}

\begin{document}
%-=-=-=-=-=-=-=-=-=-=-=-=-=-=-=-=-=-=-=-=-=-=-=-=
%
%	CHAPTER
%
%-=-=-=-=-=-=-=-=-=-=-=-=-=-=-=-=-=-=-=-=-=-=-=-=

\chapter{Optimization Problems}

%-=-=-= EXAMPLE
\begin{example}[id:20151020-171605] \label{20151020-171605}\index{Example!20151020-171605} \hfill \\
Mark has 24 meters of fencing to build a rectangular pen for his two mini pigs situation along a river.  If he is not going to place fencing along the river, then what are the dimensions of the pen that produce the largest area?

\soln

\solnsteps

Let $x$ be the width of the pen and $y$ be the length.  See the figure below:

\begin{center}
\begin{tikzpicture}
\draw[sthlmRed] (2,1) -- node[anchor=east]{$x$} (2,3) -- node[anchor=south]{$y$} (6,3) -- (6,1); 
\node (area) at (4,2) {Pen Area}; 
\fill[sthlmLightBlue] (0,0) rectangle (8,1);
\end{tikzpicture}
\end{center}

We are then able to express the area of the pen in terms of the width $x$ and the length $y$.

\begin{align*}
\text{Area} &= xy 
\end{align*}

The problem is that the area is expressed in terms of two variables $x$ and $y$.  The challenge is now to find a second relationship between these.  Since we are given the total length of fencing available, 24m, we can express the perimeter of fencing available in terms of the two variables $x$ and $y$. Furthermore, we can express $y$ in terms of only the variable $x$.

\begin{align*}
\text{Perimeter of Fencing} & = x + y + x \\
\text{Perimeter of Fencing} &= 2x + y && \text{Simplify} \text{\, goto \,} \, \ref{20151020-175447} \\
24 \unit{m} &= 2x + y && \text{SPE} \eqref{eq:spe} \\
y &=  -2x + 24 \unit{m} && \text{Solve for } y \text{\, goto \,} \, \ref{20151020-180416}
\end{align*} 

\begin{align*}
A & = xy \\
A &= x \farg{\neg 2x+24x \unit{m}} && \text{SPE} \eqref{eq:spe} \\
A &= \neg 2x^2+24x \unit{m} && \text{Simplify} \text{\, goto \,} \, \ref{20151020-214228}
\end{align*}

We are now interested in finding the maximum of the graph of the function $A(x)=\neg 2x^2+24$.  We can do this by finding if a critical point exists and subsequently showing that it can be classified as a maximum point.

\begin{align*}
\farg{A(x)}' & = \farg{\neg 2x^2+24x \unit{m}} && \text{SPE} \eqref{eq:spe} \\ 
A'(x) &= \neg 4x+ 24 \unit{m}  &&\text{Differentiating} \text{\, goto \,} \, \ref{20151011-154209} \\
\end{align*}

Find the $x$ value of the critical point by setting the derivative equal to zero and solving for $x$.

\begin{align*}
\neg 4x + 24 \unit{m} & = 0 \\
x &= 6 \unit{m}  &&\text{Solving} \text{\, goto \,} \, \ref{20151021-060347} \\
\end{align*}

There is one critical point at $x=6 \unit{m}$.  We now need to show that it is a local maximum value.  We can do this using the second derivative test (\text{\, goto \,} \ref{20151021-063855}) or the first derivative test (\text{\, goto \,}  \ref{20151021-061907}).

Now that we have shown that the area function $A(x)$ has a maximum value at the critical point $x= 6\unit{m}$, we know that the width, $x$, will be $6 \unit{m}$ when the area has reached a maximum value.  We can find the length, $y$, of the pen by using the perimeter equation that we used earlier.

\begin{align*}
y &=  \neg 2x + 24 \unit{m} \\
y &=  \neg 2 \farg{6 \unit{m}} + 24 \unit{m} && \text{SPE} \eqref{eq:spe} \\
y &=  \neg 12 \unit{m} + 24 \unit{m} && \text{OOM} \eqref{eq:oom} \\
y &=  12 \unit{m} && \text{OOA} \eqref{eq:ooa} 
\end{align*}

It might also be worth finding the maximum area of the pen.

\begin{align*}
A &= xy \\
A &= \farg{6 \unit{m}}\farg{12 \unit{m}} && \text{SPE} \eqref{eq:spe} \\
A &= 72\unit{m}^2  &&\text{Simplify} \text{\, goto \,} \, \ref{20151021-065906} \\
\end{align*}

\end{example}


\end{document}

