%!TEX root = /Users/markholson/Dropbox/+Projects/LatexFiles/MathNotebook/20150312-163541-rs2.2N-MarksMathNotebook.tex
%-=-=-=-=-=-=-=-=-=-=-=-=-=-=-=-=-=-=-=-=-=-=-=-=
%
%	CHAPTER 
%
%-=-=-=-=-=-=-=-=-=-=-=-=-=-=-=-=-=-=-=-=-=-=-=-=

\chapterimage{chapter_head_2.pdf} % Chapter heading image

\chapter{Algebraic Expressions}

%-=-=-=-=-=-=-=-=-=-=-=-=-=-=-=-=-=-=-=-=-=-=-=-=
%	SECTION: OPERATION OF ADDITION
%-=-=-=-=-=-=-=-=-=-=-=-=-=-=-=-=-=-=-=-=-=-=-=-=
\section{Expressions}\index{Algebraic Expressions}

\begin{essentialq}\hfill \\

\begin{enumerate}
	\item What is an algebraic expression?
\end{enumerate}

\end{essentialq}


\begin{table}
\begin{tabular}{|l|c|c|c|c|c|c}
\hline
 									& Arithmetic 	& Polynomial	& Algebraic  	\\
\hline
Constant 							& \cellyes		& \cellyes 		& \cellyes		\\
\hline
Factorial 							& \cellyes 		& \cellyes 		& \cellyes		\\
\hline
Variable: parameter/coefficient 	& \cellyes 		& \cellyes 		& \cellyes		\\
\hline
Variable: unknown/indeterminate 	& \cellno 		& \cellyes 		& \cellyes		\\
\hline
Power with $\integer^{+}$ exponent 	& \cellno 		& \cellyes 		& \cellyes		 \\
\hline
Power with $\integer$ exponent 		& \cellno 		& \cellno 		& \cellyes		 \\
\hline
$n$-th root 						& \cellno 		& \cellno 		& \cellyes		\\
\hline
Power with $\mathbb{Q}$ exponent 	& \cellno 		& \cellno 		& \cellyes		 \\
\hline
\end{tabular}
\caption{Names of different types of expressions}\label{tab:tableofexpressions}
\end{table}


%-=-=-= DEFINITION
%\begin{definition}[Algebraic Expression]\index{Algebraic Expression}

%An algebraic expression is made up of one or more terms (operation of addition), which are themselves made up of factors (operation of multiplication).\\

%Algebraic expressions have three different types of multiplicative factors:\\

%\begin{enumerate}   
%	\item constant (coefficient) 
%	\begin{itemize}
%		\item real number 
%		\item parameter 
%	\end{itemize}
%	\item variable
%	\item rational power
%\end{enumerate}
%\hfill \cite{mathworld:algebraicexpression}
%\end{definition}

\section{Polynomial Expressions}\index{Polynomial Expressions}

%-=-=-= DEFINITION
\begin{definition}[Operation of Addition (OOA)]\index{Operation!Operation of Addition}
\begin{align}
\underbrace{\underbrace{a}_{\text{Augend}}+\underbrace{b}_{\text{Addend}}}_{\text{Sum}} \label{eq:ooa}
\end{align}
\end{definition}


%-=-=-= DEFINITION
\begin{definition}[Common Denominator (CD)]\index{Common Denominator}
\begin{subequations}
\begin{align}
\dfrac{a}{b} + \dfrac{c}{b} &= \dfrac{a+c}{b} \label{eq:cd1} \\
\dfrac{a+c}{b}&= \dfrac{a}{b} + \dfrac{c}{b} \label{eq:cd2}
\end{align}
\end{subequations}
\end{definition}

%-=-=-= RULE
\begin{arule}[Fraction Operation of Addition (FOOA)]\index{Fraction Operation of Addition}
\begin{subequations}
\begin{align}
\dfrac{a}{b} + \dfrac{c}{d} &= \dfrac{ad+bc}{bd} \label{eq:fooa1} \\
\dfrac{ad+bc}{bd} &= \dfrac{a}{b} + \dfrac{c}{d} \label{eq:fooa2}
\end{align}
\end{subequations}
\end{arule}

\newpage
\begin{figure}
\begin{tikzpicture}[scale=1, node distance=1.1cm, text width=5em, auto]
    % Place nodes
\node [property] (MId1) {MId};
\node [notation, below of=MId1] (ONeg1) {ONeg};
\node [operation, below of=ONeg1] (DOS1) {DOS};
\node [delim, below of=DOS1] (DELIM) {DELIM};
\node [property, below of=DELIM] (DPE) {DPE};
\node [notation, below of=DPE] (JTC) {JTC};
\node [property, below of=JTC] (CPM) {CPM};
\node [operation, below of=CPM] (OOM) {OOM};
\node [property, right of=OOM, node distance=2.8cm] (APM) {APM};
\node [algorithm, below of=OOM] (RF1) {RF};
\node [algorithm, below of=RF1] (CTJ) {CTJ};
\node [notation, below of=CTJ] (CPA) {CPA};
\node [property, below of=CPA] (DPF) {DPF};
\node [operation, below of=DPF] (OOA) {OOA};
\node [property, right of=OOA,node distance=2.8cm] (APA) {APA};
\node [algorithm, below of=OOA] (RF2) {RF};
\node [property, below of=RF2] (AId) {AId};
\node [operation, below of=AId] (DOS2) {DOS};
\node [notation, below of=DOS2] (ONeg2) {ONeg};
\node [property, below of=ONeg2] (MId2) {MId};

\path [line] (MId1) -- (ONeg1);
\path [line] (ONeg1) -- (DOS1);
\path [line] (DOS1) -- (DELIM);
\path [line] (DELIM) -- (DPE);
\path [line] (DPE) -- (JTC);
\path [line] (JTC) -- (CPM);
\path [line] (CPM) edge [out= 315, in= 150] (APM);
\path [line] (APM) -- (OOM);
\path [line] (OOM) -- (RF1);
\path [line] (RF1) -- (CTJ);
\path [line] (CTJ) -- (CPA);
\path [line] (CPA) -- (DPF);
\path [line] (DPF) edge [out= 315, in= 150] (APA);
\path [line] (APA) -- (OOA);
\path [line] (OOA) -- (RF2);
\path [line] (RF2) -- (AId);
\path [line] (AId) -- (DOS2);
\path [line] (DOS2) -- (ONeg2);
\path [line] (ONeg2) -- (MId2);

\node [function, right of=DELIM, node distance=8cm] (PoTF) {PoTF};
\node [function, below of=PoTF] (RTPo) {RTPo};
\node [function, below of=RTPo] (PoNE1) {PoNE};
\node [function, below of=PoNE1] (PoPr) {PoPr};
\node [function, below of=PoPr] (PoQ) {PoQ};
\node [function, below of=PoQ] (PoPrPo) {PoPrPo};
\node [function, below of=PoPrPo] (PoQPo) {PoQPo};
\node [function, below of=PoQPo] (PrCBPo) {PrCBPo};
\node [function, below of=PrCBPo] (QCBPo) {QCBPo};
\node [function, below of=QCBPo] (PoPo) {PoPo};
\node [function, below of=PoPo] (PoNE2) {PoNE};
\node [operation, below of=PoNE2] (OOE) {OOE};
\node [function, below of=OOE] (PoTR) {PoTR};


\path [line] (DELIM) edge [bend left=30] (PoTF);
\path [line] (PoTF) -- (RTPo);
\path [line] (RTPo) -- (PoNE1);
\path [line] (PoPr) -- (PoQ);
\path [line] (PoQ) -- (PoPrPo);
\path [line] (PoPrPo) -- (PoQPo);
\path [line] (PoQPo) -- (PrCBPo);
\path [line] (PrCBPo) -- (QCBPo);
\path [line] (QCBPo) -- (PoPo);
\path [line] (PoPo) -- (PoNE2);
\path [line] (PoNE2) -- (OOE);
\path [line] (OOE) -- (PoTR);

%\node[productdotted, fit =(DELIM)(JTC)(CPM)(OOM)(RF1)(CTJ)(APM)](product) {};
%\node[left of =product, text centered,](powers) {\textcolor{sthlmPurple}{\textbf{Multiplication}}};
%\node[sumdotted, fit =(CPA)(DPF)(OOA)(RF2)(AId)(ONeg2)(DOS2)(MId2)(APA)](sum2) {};
%\node[sumdotted, fit =(MId1)(ONeg1)(DOS1)](sum1) {};
%\node[left of =sum1, text centered,](powers) {\textcolor{sthlmPurple}{\textbf{Addition}}};
%\node[left of =sum2, text centered,](powers) {\textcolor{sthlmPurple}{\textbf{Addition}}};


\node[functiondotted, fit =(PoTF)(RTPo)(PoNE1)(PoPr)(PoQ)(PoPrPo)(PoQPo)(PrCBPo)(QCBPo)(PoPo)(PoNE2)(OOE)(PoTR)](powerfunctions) {};
\node[below of =PoTR, text centered,](powers) {\textcolor{sthlmPurple}{\textbf{Powers}}};

\path [line] (PoNE1) edge [out= 180, in= 0] (DELIM);

\path [line] (CPM) edge [out= 0, in= 180] (PoPr);

\path [line] (PoTR) edge [out= 180, in= 0] (APM);

\path [line] (AId) edge [out= 180, in= 180] (DELIM);

\end{tikzpicture}
\caption{Simplifying Expressions Workflow: \newline {\color{stockholmPink!40}\ensuremath{\blacksquare}} Property, {\color{stockholmBlue!40}\ensuremath{\blacksquare}} Operation, {\color{stockholmGreen!40}\ensuremath{\blacksquare}} Notation, {\color{stockholmPurple!40}\ensuremath{\blacksquare}} Powers, {\color{stockholmYellow!40}\ensuremath{\blacksquare}} Delimiters, {\color{stockholmOrange!40}\ensuremath{\blacksquare}} Process, {\color{stockholmDarkGrey!40}\ensuremath{\blacksquare}} Not Used}
\end{figure}


%-=-=-= DEFINITION
\begin{definition}[Operation of Multiplication (OOM)] \index{Operation!Operation of Multiplication}
\begin{align}
\underbrace{\underbrace{a}_{\text{Multiplicand}} \cdot \underbrace{b}_{\text{Multiplier}}}_{\text{Product}} \label{eq:oom}
\end{align}
\end{definition}



%-=-=-= DEFINITION
\begin{definition}[Operation of Exponentiation (OOE)]\index{Operation!Operation of Exponentiation}
\begin{align}
\underbrace{{\underbrace{b}_{base}}^{\overbrace{m}^{Exponent}}}_{Power} \label{eq:ooe}
\end{align}
\end{definition}


%-=-=-=-=-=-=-=-=-=-=-=-=-=-=-=-=-=-=-=-=-=-=-=-=
%	SECTION: ALGEBRAIC EXPRESSIONS
%-=-=-=-=-=-=-=-=-=-=-=-=-=-=-=-=-=-=-=-=-=-=-=-=






%-=-=-= DEFINITION
\begin{definition}[Juxtaposition to Center-Dot (JTC)]\index{Notation!Juxtapostion to Center-Dot}
\begin{align}
ab &= a \cdot b \label{eq:jtc}
\end{align}
\end{definition}

%-=-=-= DEFINITION
\begin{definition}[Center-Dot to Justapostion (CTJ)]\index{Notation!Center-Dot to Juxtaposition}
\begin{align}
a \cdot b &= ab \label{eq:ctj}
\end{align}
\end{definition}


%-=-=-= DEFINITION
\begin{definition}[Commutative Property of Multiplication (CPM)]\index{Property!Commutative Property of Multiplication}
\begin{align}
\alert{a} \cdot b &= b \cdot \alert{a} \label{eq:cpm}
\end{align}
\end{definition}



%-=-=-= DEFINITION
\begin{definition}[Multiplicative Inverse (MI)]\index{Multiplicative Inverse}
\begin{subequations}
\begin{align}
a \cdot \alert{\dfrac{1}{a}} &= 1 \label{eq:mi1} \\
a \cdot \alert{a^{-1}} &= 1 \label{eq:mi2} 
\end{align}
\end{subequations}
\end{definition}

%-=-=-= DEFINITION
\begin{definition}[Associative Property of Multiplication (APM)]\index{Property!Associative Property of Multiplication}
\begin{subequations}
\begin{align}
a\cdot b\cdot c &= (a\cdot b)\cdot c \label{eq:apm1} \\
a\cdot b\cdot c &= a\cdot (b\cdot c) \label{eq:apm2}
\end{align}
\end{subequations}
\end{definition}


%-=-=-=-=-=-=-=-=-=-=-=-=-=-=-=-=-=-=-=-=-=-=-=-=
%	SECTION: Powers
%-=-=-=-=-=-=-=-=-=-=-=-=-=-=-=-=-=-=-=-=-=-=-=-=

\subsubsection{Powers}\index{Powers}


\begin{arule}[Power of a Quotient of Powers (PoQPo)]\index{Powers!Power of a Quotient of Powers}
\begin{subequations}
\begin{align}
	\left(\dfrac{a^m}{b^n}\right)^k &= \dfrac{a^{m \cdot k}}{b^{n\cdot k}} \label{eq:poqpo1}\\
	\dfrac{a^{m \cdot k}}{b^{n\cdot k}} &= \left(\dfrac{a^m}{b^n}\right)^k \label{eq:poqpo2}
\end{align}
\end{subequations}
\end{arule}

\begin{arule}[Power of a Product of Powers (PoPrPo)]\index{Powers!Power of a Product of Powers}
\begin{subequations}
\begin{align}
	\left(a^m \cdot b^n\right)^k &= a^{m \cdot k} \cdot b^{n\cdot k} \label{eq:poprpo1}\\
	a^{m \cdot k} \cdot b^{n \cdot k} &= \left(a^m \cdot b^n\right)^k \label{eq:poprpo2}
\end{align}
\end{subequations}
\end{arule}

\begin{definition}[Power To Factor (PoTF)]\index{Powers!Power to Factor}
\begin{align}
a^{\alert{n}} &= a_1 \cdot a_2 \cdot \ldots \cdot a_{n-1} \cdot a_{\alert{n}} \label{eq:potf} 
\end{align}
\end{definition}

\begin{definition}[Factor To Power (FTPo)]\index{Powers!Factor to Power}
\begin{align}
a_1 \cdot a_2 \cdot \ldots \cdot a_{n-1} \cdot a_{\alert{n}} &= a^{\alert{n}} \label{eq:ftpo} 
\end{align}
\end{definition}

%-=-=-= DEFINITION
\begin{definition}[Power Inverse (PoI)]\index{Power Inverse}
\begin{subequations}
\begin{align}
\left(b^{m}\right)^{\frac{1}{m}} &= b  \label{eq:poi} 
\end{align}
\end{subequations}
\end{definition}

%-=-=-= DEFINITION
\begin{definition}[Power Inverse (PoId)]\index{Power Identity}
\begin{subequations}
\begin{align}
1&= b^0  \label{eq:poid1} \\
b^{0}&= 1  \label{eq:poid2} 
\end{align}
\end{subequations}
\end{definition}

\begin{notation}[Radical To Power (RTPo)]\index{Powers!Radical to Power}
\begin{align}
\sqrt[m]{b^n} &= b^{\frac{n}{m}}	 \label{eq:rtpo} 
\end{align}
\end{notation}

\begin{notation}[Power To Radical (PoTR)]\index{Powers!Power to Radical}
\begin{align}
b^{\frac{n}{m}} &= \sqrt[m]{b^n}	 \label{eq:potr} 
\end{align}
\end{notation}




%-=-=-=-=-=-=-=-=-=-=-=-=-=-=-=-=-=-=-=-=-=-=-=-=
%	SUBSECTION: Monomials of Like Terms
%-=-=-=-=-=-=-=-=-=-=-=-=-=-=-=-=-=-=-=-=-=-=-=-=

\subsection{Monomials of Like Terms}\index{Monomials of Like Terms}








%-=-=-= DEFINITION
\begin{definition}[Additive Inverse (AI)]\index{Additive Inverse}
\begin{subequations}
\begin{align}
a + \alert{(-a)} &= 0 \label{eq:ai}
\end{align}
\end{subequations}
\end{definition}




\subsection{Surds}\index{Surds}

%-=-=-= EXAMPLE
\begin{example}[id:20141108-085327] \label{20141108-085327} \index{Example!20141108-085327} \hfill \\

Simplify $2\sqrt{2}-\dfrac{\left(\sqrt{2}\right)^3}{3}-\left(2\left(-\sqrt{2}\right)-\dfrac{\left(-\sqrt{2}\right)^3}{3} \right)$

\soln

\solnsteps
\begin{align*}
2 \sqrt{2}- \dfrac{\alert{1}\left(\alert{1}\sqrt{2}\right)^{3}}{3}-\alert{1} \left(2 \left(-\alert{1}\sqrt{2} \right) - \dfrac{\alert{1} \left(-\alert{1} \sqrt{2}\right)^{3}}{3} \right) && \text{MId} \eqref{eq:mid1} \\
2 \sqrt{2} - \dfrac{1 \left(1\sqrt{2}\right)^{3}}{3}-1 \left(2 \left(\alert{\neg 1} \sqrt{2} \right) - \dfrac{1 \left(\alert{\neg 1} \sqrt{2}\right)^{3}}{3} \right) && \text{ONeg} \eqref{eq:oneg1} \\
2 \sqrt{2} + \dfrac{\neg 1 \left(1\sqrt{2}\right)^{3}}{3}+ \neg 1 \left(2 \left(\alert{\neg 1} \sqrt{2} \right) + \dfrac{\neg 1 \left(\alert{\neg 1} \sqrt{2}\right)^{3}}{3} \right) && \text{DOS} \eqref{eq:dos1} \\
2 \cdot \alert{2^{1/2}} + \dfrac{\neg 1 \left(1 \cdot \alert{2^{1/2}}\right)^{3}}{3} + \neg 1 \left(2 \left(\neg  1 \cdot \alert{2^{1/2}} \right) + \dfrac{\neg 1 \left(\neg  1  \cdot \alert{2^{1/2}}\right)^{3}}{3} \right) && \text{RTPo} \eqref{eq:rtpo} \\
2 \cdot 2^{1/2} + \dfrac{\neg 1 \left(1 \cdot 2^{1/2}\right)^{3}}{3} + \neg 1 \left(\alert{2 \cdot \neg  1 \cdot 2^{1/2}} + \dfrac{\neg 1 \left(\neg 1  \cdot 2^{1/2}\right)^{3}}{3} \right) && \text{JTC} \eqref{eq:jtc} \\
2 \cdot 2^{1/2} + \dfrac{\neg 1 \cdot \alert{1 \cdot 2^{3/2}}}{3} + \neg 1 \left(2 \cdot \neg  1 \cdot 2^{1/2} + \dfrac{\neg 1 \cdot \alert{\neg 1  \cdot 2^{3/2}}}{3} \right) && \text{PoPrPo} \eqref{eq:poprpo1} \\
2 \cdot 2^{1/2} + \dfrac{\neg 1 \cdot 1 \cdot \alert{2^{2/2} \cdot 2^{1/2}}}{3} + \neg 1 \left(2 \cdot \neg  1 \cdot 2^{1/2} + \dfrac{\neg 1 \cdot \neg 1  \cdot \alert{2^{2/2} \cdot 2^{1/2}}}{3} \right) && \text{PrCBPo} \eqref{eq:prcbpo2} \\
2 \cdot 2^{1/2} + \dfrac{\neg 1 \cdot 1 \cdot \alert{2} \cdot 2^{1/2}}{3} + \neg 1 \left(2 \cdot \neg  1 \cdot 2^{1/2} + \dfrac{\neg 1 \cdot \neg 1  \cdot \alert{2} \cdot 2^{1/2}}{3} \right) && \text{MId} \eqref{eq:mid2} \\
2 \cdot \alert{\sqrt{2}} + \dfrac{\neg 1 \cdot 1 \cdot 2 \cdot \alert{\sqrt{2}}}{3} + \neg 1 \left(2 \cdot \neg  1 \cdot \alert{\sqrt{2}} + \dfrac{\neg 1 \cdot \neg 1  \cdot 2 \cdot \alert{\sqrt{2}}}{3} \right) && \text{PoTR} \eqref{eq:potr} \\
2 \cdot \sqrt{2} + \dfrac{\neg 1 \cdot 1 \cdot 2 \cdot \sqrt{2}}{3} + \alert{\neg 1} \cdot 2 \cdot \neg  1 \cdot \sqrt{2} + \dfrac{\alert{\neg 1} \cdot \neg 1 \cdot \neg 1  \cdot 2 \cdot \sqrt{2}}{3}  && \text{DPE} \eqref{eq:dpe1} \\
2 \cdot \sqrt{2} + \dfrac{\neg 2 \cdot \sqrt{2}}{3} + 2 \cdot \sqrt{2} + \dfrac{\neg 2 \cdot \sqrt{2}}{3}  && \text{OOM} \eqref{eq:oom} \\
2\sqrt{2} + \dfrac{\neg 2\sqrt{2}}{3} + 2\sqrt{2} + \dfrac{\neg 2\sqrt{2}}{3} && \text{CTJ} \eqref{eq:ctj} \\
\left(2 + \dfrac{\neg 2}{3} + 2 + \dfrac{\neg 2}{3} \right)  \sqrt{2} && \text{DPF} \eqref{eq:dpf1} \\
\dfrac{8}{3}\sqrt{2} && \text{OOA} \eqref{eq:ooa} 
\end{align*}

\qdepend 

\qdependlist
example \ref{20141108-083108}-20141108-083108

\end{example}



