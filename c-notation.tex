\documentclass[20150903-160354-rs2.2-MarksMathNotebook.tex]{subfiles}

\begin{document}
%-=-=-=-=-=-=-=-=-=-=-=-=-=-=-=-=-=-=-=-=-=-=-=-=
%
%	CHAPTER
%
%-=-=-=-=-=-=-=-=-=-=-=-=-=-=-=-=-=-=-=-=-=-=-=-=

\chapter{Notation}

\section{Negation Notation}\index{Notation!Negation Notation}
%-=-=-= NOTATION
\begin{notations}[Operation of Negation (ONeg)]\index{Operation!Operation of Negation}

\begin{subequations}
\begin{align}
-a &= \neg a \label{eq:oneg1} \\
\neg a &= -a \label{eq:oneg2}
\end{align}
\end{subequations}

I have used a different symbol, $\neg$, as the prefix negation operator only to differentiate it from the minus sign infix operator symbol, $-$, which is also used as the infix operator for the dyadic operation of subtraction.  I will refer to this change of symbol as ONeg.  This is used only as a teaching tool and should not be confused with the logic negation operator.  Another advantage of using this symbol is that it reduces the number of delimiters used in an expression for example, $\neg a$ versus $(-a)$.

\begin{itemize}
	\item Negative five: $-5$
	\item Negative five: $\neg 5$
	\item Four minus five: $4-5$
	\item Four minus negative five: $4--5$
	\item Four minus negative five: $4-(-5)$
	\item Four minus negative five: $4-\neg 5$
	\item Negative four minus five: $-4-5$
	\item Negative four minus five: $\neg 4-5$
\end{itemize}

\end{notations}

%-=-=-=-=-=-=-=-=-=-=-=-=-=-=-=-=-=-=-=-=-=-=-=-=
%	SECTION:
%-=-=-=-=-=-=-=-=-=-=-=-=-=-=-=-=-=-=-=-=-=-=-=-=

\section{Multiplication Notation}\index{Notation!Multiplication Notation}

%-=-=-= NOTATION
\begin{notations}[Multiplication Center-Dot (MC)]\index{Notation!Multiplication Center-Dot}
\begin{align}
a\cdot b \label{eq:mc}
\end{align}
\end{notations}

%-=-=-= NOTATION
\begin{notations}[Multiplication Juxtaposition (MJ)]\index{Notation!Multiplication Juxtaposition}
\begin{align}
ab, a(b), (a)b, (a)(b), a[b], [a]b,[a][b]\label{eq:mj}
\end{align}
\end{notations}

%-=-=-= NOTATION
\begin{notations}[Multiplication Times (MT)]\index{Notation!Multiplication Times}
\begin{align}
a \times b\label{eq:mt}
\end{align}
\end{notations}

%-=-=-= NOTATION
\begin{notations}[Juxtaposition to Center-Dot (JTC)]\index{Notation!Juxtapostion to Center-Dot}
\begin{align}
ab &= a \cdot b \label{eq:jtc}
\end{align}
\end{notations}

%-=-=-= NOTATION
\begin{notations}[Center-Dot to Justapostion (CTJ)]\index{Notation!Center-Dot to Juxtaposition}
\begin{align}
a \cdot b &= ab \label{eq:ctj}
\end{align}
\end{notations}

%-=-=-=-=-=-=-=-=-=-=-=-=-=-=-=-=-=-=-=-=-=-=-=-=
%	SECTION:
%-=-=-=-=-=-=-=-=-=-=-=-=-=-=-=-=-=-=-=-=-=-=-=-=

\section{Power Notation}\index{Notation!Power Notation}

%-=-=-= NOTATION
\begin{notations}[Power Exponent Negative Exponent (PoNegE)]\index{Powers!Power Negative Exponent}
\begin{align}
b^{-k}&= \frac{1}{b^k} \label{eq:pone1}\\
\frac{1}{b^k}&=b^{-k}  \label{eq:pone2}
\end{align}
\end{notations}

%-=-=-= NOTATION
\begin{notations}[Power To Factor (PoTF)]\index{Powers!Power to Factor}
\begin{align}
a^{\alert{n}} &= a_1 \cdot a_2 \cdot \ldots \cdot a_{n-1} \cdot a_{\alert{n}} \label{eq:potf}
\end{align}
\end{notations}

%-=-=-= NOTATION
\begin{notations}[Power To Logarithm (PoTL)]\index{Powers!Power To Logarithm}
\begin{align}
y=b^x \Rightarrow x=\log_b y \label{eq:potl}
\end{align}
\end{notations}

%-=-=-= NOTATION
\begin{notations}[Factor To Power (FTPo)]\index{Powers!Factor to Power}
\begin{align}
a_1 \cdot a_2 \cdot \ldots \cdot a_{n-1} \cdot a_{\alert{n}} &= a^{\alert{n}} \label{eq:ftpo}
\end{align}
\end{notations}

%-=-=-= NOTATION
\begin{notations}[Radical To Power (RTPo)]\index{Powers!Radical to Power}
\begin{align}
\sqrt[m]{b^n} &= b^{\frac{n}{m}}	 \label{eq:rtpo}
\end{align}
\end{notations}

%-=-=-=-=-=-=-=-=-=-=-=-=-=-=-=-=-=-=-=-=-=-=-=-=
%	SECTION:
%-=-=-=-=-=-=-=-=-=-=-=-=-=-=-=-=-=-=-=-=-=-=-=-=

\section{Logarithm Notation}\index{Notation!Logarithm Notation}

%-=-=-= NOTATION
\begin{notations}[Logarithm Exponent Visible (LEV)]\index{Logartihms!Logarithm Exponent Visible}
\begin{align}
\log_b y \Rightarrow \log_b y=x	 \label{eq:lev}
\end{align}
\end{notations}

%-=-=-= NOTATION
\begin{notations}[Logarithm to Power (LTPo)]\index{Logartihms!Logarithm to Power}
\begin{align}
x =\log_b y \Rightarrow y=b^x	 \label{eq:ltpo}
\end{align}
\end{notations}

%-=-=-=-=-=-=-=-=-=-=-=-=-=-=-=-=-=-=-=-=-=-=-=-=
%	SECTION:
%-=-=-=-=-=-=-=-=-=-=-=-=-=-=-=-=-=-=-=-=-=-=-=-=

\section{Derivative Notation}\index{Notation!Derivative Notation}


%-=-=-= NOTATION
\begin{notations}[Leibniz's first derivative]\index{Notation!Leibniz's first derivative}
\begin{align}
\dfrac{\dy}{\dx} = \dfrac{d \left[f(x) \right]}{\dx}= \dfrac{d}{\dx} \left[f(x) \right]	 \label{eq:leibfirstd}
\end{align}
\end{notations}

%-=-=-= NOTATION
\begin{notations}[Leibniz's second derivative]\index{Notation!Leibniz's second derivative}
\begin{align}
\dfrac{\dd^{2}y}{\dx^{2}}	 \label{eq:leibsecondd}
\end{align}
\end{notations}

%-=-=-= NOTATION
\begin{notations}[Leibniz's nth derivative]\index{Notation!Leibniz's nth derivative}
\begin{align}
\dfrac{\dd^{n}y}{\dx^{n}}	 \label{eq:leibnd}
\end{align}
\end{notations}

%-=-=-= NOTATION
\begin{notations}[Leibniz's Evaluate derivative]\index{Notation!Leibniz's Evaluate derivative}
\begin{align}
\dydx\left.{\!\!\frac{}{}}\right|_{x=a} = \dydx(a)	 \label{eq:leibEvalD}
\end{align}
\end{notations}

%-=-=-= NOTATION
\begin{notations}[LaGrange's first derivative]\index{Notation!LaGrange's first derivative}
\begin{align}
f'(x)	 \label{eq:lagfirstd}
\end{align}
\end{notations}

%-=-=-= NOTATION
\begin{notations}[LaGrange's second derivative]\index{Notation!LaGrange's second derivative}
\begin{align}
f''(x)	 \label{eq:lagsecondd}
\end{align}
\end{notations}

%-=-=-= NOTATION
\begin{notations}[LaGrange's nth derivative]\index{Notation!LaGrange's nth derivative}
\begin{align}
f^{(n)}(x)	 \label{eq:lagnd}
\end{align}
\end{notations}

%-=-=-= NOTATION
\begin{notations}[LaGrange's Evaluate derivative]\index{Notation!LaGrange's Evaluate derivative}
\begin{align}
f'(a)	 \label{eq:lagEvalD}
\end{align}
\end{notations}



%-=-=-= NOTATION
\begin{notations}[Euler's first derivative]\index{Notation!Euler's first derivative}
\begin{align}
Df =D_{x}f	 \label{eq:eulfirstd}
\end{align}
\end{notations}

%-=-=-= NOTATION
\begin{notations}[Euler's second derivative]\index{Notation!Euler's second derivative}
\begin{align}
D^{2}f=D_{x}^{2}f	 \label{eq:eulsecondd}
\end{align}
\end{notations}

%-=-=-= NOTATION
\begin{notations}[Euler's nth derivative]\index{Notation!Euler's nth derivative}
\begin{align}
D^{n}f= D_{x}^{n}	 \label{eq:eulnd}
\end{align}
\end{notations}


\end{document}

