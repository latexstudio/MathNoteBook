%!TEX root = /Users/markholson/Dropbox/+Projects/LatexFiles/MathNotebook/20150312-163541-rs2.2N-MarksMathNotebook.tex
%-=-=-=-=-=-=-=-=-=-=-=-=-=-=-=-=-=-=-=-=-=-=-=-=
%
%	CHAPTER 
%
%-=-=-=-=-=-=-=-=-=-=-=-=-=-=-=-=-=-=-=-=-=-=-=-=

\chapterimage{chapter_head_2.pdf} % Chapter heading image

\chapter{Functions}

%-=-=-=-=-=-=-=-=-=-=-=-=-=-=-=-=-=-=-=-=-=-=-=-=
%	SECTION: Evaluating Functions
%-=-=-=-=-=-=-=-=-=-=-=-=-=-=-=-=-=-=-=-=-=-=-=-=

\section{Evaluating Functions}\label{Evaluating Functions}

\begin{example}[id:20141106-083703]\label{20141106-083703}\index{Example!20141106-083703} \hfill \\
Given $f(x)= \dfrac{1}{12}x^3-x^2+4x$, evaluate $f(3)-f(2)$.

\soln

\solnsteps
\begin{align*}
f(3)-f(2) & = \left(\dfrac{\farg{3}^3}{12} - \farg{3}^2 + 4\farg{3} \right) - \left(\dfrac{\farg{2}^3}{12} - \farg{2}^2 + 4\farg{2} \right) \\
&= \left(\dfrac{(3)^3}{12} - (3)^2 + 4(3) \right) - 1 \left(\dfrac{(2)^3}{12} - (2)^2 + 4(2) \right) && \text{MId} \eqref{eq:mid1} \\
&= \left(\dfrac{(3)^3}{12} + \neg  (3)^2 + 4(3)\right) + \neg 1 \left(\dfrac{(2)^3}{12} + \neg (2)^2 + 4(2) \right) && \text{DOS} \eqref{eq:dos1} \\
&= \left(\dfrac{27}{12} + \neg 9 + 4(3) \right)+ \neg 1 \left(\dfrac{8}{12} + \neg 4 + 4(2) \right) && \text{OOE} \eqref{eq:ooe} \\
&= \left(\dfrac{27}{12} + \neg 9 + 12 \right) + \neg 1 \left(\dfrac{8}{12} + \neg 4 + 8 \right) && \text{OOM} \eqref{eq:oom} \\
&= \left(\dfrac{27}{12} + 3 \right) + \neg 1 \left(\dfrac{8}{12} + 4 \right) && \text{OOA} \eqref{eq:ooa} \\
&= \left(\dfrac{27+36}{12} \right) + \neg 1 \left(\dfrac{8+48}{12}\right) && \text{FOOA} \eqref{eq:fooa1} \\
&= \left(\dfrac{63}{12} \right) + \neg 1 \left(\dfrac{56}{12}\right) && \text{OOA} \eqref{eq:ooa} \\
&= \left(\dfrac{63}{12} \right) + \neg \dfrac{56}{12} && \text{OOM} \eqref{eq:oom} \\
&= \dfrac{7}{12} && \text{OOA} \eqref{eq:ooa} 
\end{align*}

\qdepend 

\qdependlist

example \ref{20141105-144223}-20141105-144223


\end{example}

%-=-=-=-=-=-=-=-=-=-=-=-=-=-=-=-=-=-=-=-=-=-=-=-=
%	SECTION: Rational Functions
%-=-=-=-=-=-=-=-=-=-=-=-=-=-=-=-=-=-=-=-=-=-=-=-=

\section{Quadratic Functions}\label{Quadratic Functions}

\subsection{Completing the Square}

\begin{definition}[Completing The Square]
Completing the square is the process used to convert a quadratic polynomial function \(f(x)=ax^2+bx+c\) to the form

\[ f(x)=a \left(x+\dfrac{b}{2a} \right)^2 + \left(c-\dfrac{b^2}{4a} \right) \]

We can simplify this form by defining \(B = \dfrac{b}{2a} \) and \(C=c-\dfrac{b^2}{4a} \), which gives us 

\[ f(x)=a \left(x+B \right) + C \] 

\hfill \cite{mathworld:completethesquare}
\end{definition}

\begin{align*}
f(x) 	&= ax^2+bx+c \\
		&= a \left[x^2+ \dfrac{b}{a}x + \dfrac{c}{a} \right] && \text{DPF} \eqref{eq:dpf2} \\
		&= a \left[x^2 + \dfrac{b}{a}x + \cRed{k} + \cRed{\neg k} + \dfrac{c}{a} \right] && \text{AId} \eqref{eq:aid2} \\
		&= a \left[ \left(x^2 + \dfrac{b}{a}x + \cRed{k} \right)+ \left(\cRed{-k}+ \dfrac{c}{a}  \right) \right] && \text{APA} \eqref{eq:apa2} \\
\end{align*}

\begin{figure}[h!]
\centering
\begin{tikzpicture}[scale=1, auto]

% Place nodes

\node[firstterm](11){$x$}; \node[factoradd,right=of 11](plus1){}; \node[secondterm, right=of plus1](12){$\frac{b}{2a}$};
\node[firstterm, below=of 11](21){$x$}; \node[factoradd,right=of 21](plus2){}; \node[secondterm, right=of plus2](22){$\frac{b}{2a}$};

\node[multiply, below=of 21](31){$x^2$};
\node[multiply, below=of 22](32){$\cRed{k}=\frac{b^2}{4a^2}$};

\node[multiply, right=of 12](13){$\frac{b}{2a}x$};
\node[multiply, right=of 22](23){$\frac{b}{2a}x$};

\node[add, below=of 23](33){$\frac{b}{2}x$};

\path [line](11) edge[bend right=30]node[color=black, midway, left]{$\times$}(21);
\path [line](12) edge[bend left=30]node[color=black,, midway, right]{$\times$}(22);
\path [line](21)--(31);

\path [line](21) edge[bend left=30]node[color=black, pos=0.4, below]{$\times$}(12);
\path [line](11) edge[bend right=30](22);
\path [line](22)--(32);

\path [line](12)--(13);
\path [line](22)--(23);

\path [line](13)--node[color=black, midway, right]{$+$}(23);
\path [line](23)--(33);

\end{tikzpicture}
\caption{Factoring Organizer used to find the value of $\cRed{k}$}
\end{figure}

\begin{align*}
f(x) 	&=  a \left[ \left(x^2 + \dfrac{b}{a}x + \cRed{\dfrac{b^2}{4a^2}} \right)+ \left(\cRed{\dfrac{-b^2}{4a^2}}+ \dfrac{c}{a}  \right) \right]\\
		&= a \left[ \left(x+ \dfrac{b}{2a} \right)^2 + \cRed{\dfrac{-b^2}{4a^2}}+ \dfrac{c}{a} \right]  && \text{DPF} \eqref{eq:dpf2} \\
		&= a \left[  \left(x+ \dfrac{b}{2a} \right)^2 + \dfrac{-b^2}{4a^2}+ \dfrac{4ac}{4a^2} \right] \\ % TODO Common Denominator Step
		&= a \left[  \left(x+ \dfrac{b}{2a} \right)^2 + \dfrac{4ac-b^2}{4a^2} \right]  && \text{OOA} \eqref{eq:ooa} \\
		&= a \left(x+ \dfrac{b}{2a} \right)^2 + \dfrac{4ac-b^2}{4a} && \text{DPE} \eqref{eq:dpe1} \\ 
		&= a \left(x+ \dfrac{b}{2a} \right)^2 + \left(c-\dfrac{b^2}{4a} \right) % TODO Reduce fraction step 
\end{align*}


%-=-=-=-=-=-=-=-=-=-=-=-=-=-=-=-=-=-=-=-=-=-=-=-=
%	SECTION: Trigonometric Functions
%-=-=-=-=-=-=-=-=-=-=-=-=-=-=-=-=-=-=-=-=-=-=-=-=

\section{Trigonometric Functions}\label{Trigonometric Functions}

\subsection{Graphing Trigonometric Function}\label{Graphing Trigonometric Functions}

\newpage
\subsubsection{Graphing the Sine Function}\label{Graphing the Sine Function}

\begin{center}
\begin{tikzpicture}[scale=1, auto]
% Place nodes
\node[task](1){Draw principal axis $y_{sa}=D$ };
\node[decision, right= of 1] (2) {$B<0$};
\node[task, right= of 2] (3) {$(-Bx \pm C)=-1(Bx\mp C)$};
\node[task, below= of 3] (4) {$A \sin [-1(Bx\mp C)]= -A \sin(Bx \mp C)$};
\node[task, left= of 4] (5) {$P= \dfrac{360 \degree}{ \lvert B \rvert}$};
\node[task, left= of 5] (6) {$PS = -\dfrac{C}{B}$};

\node[abscissa, below= of 5, node distance=2.0cm] (7) {$x_0= PS$};
\node[abscissa, below= of 7] (8) {$x_1 = PS + \dfrac{P}{4}$};
\node[abscissa, below= of 8] (9) {$x_2=PS + \dfrac{P}{2}$};
\node[abscissa, below= of 9] (10) {$x_3= PS + \dfrac{3P}{4}$};
\node[abscissa, below= of 10] (11) {$x_4= PS + P$};

\node[task, below= of 11] (12) {$ a=\lvert A \rvert$};
\node[decision, below= of 12] (13) {$A>0$};

\node[ordinate, left= of 11] (14) {$y_4=D$};
\node[ordinate, above= of 14] (15) {$y_3=D-a$};
\node[ordinate, above= of 15] (16) {$y_2=D$};
\node[ordinate, above= of 16] (17) {$y_1=D+a$};
\node[ordinate, above= of 17] (18) {$y_0=D$};

\node[ordinate, right= of 11] (19) {$y_4=D$};
\node[ordinate, above= of 19] (20) {$y_3=D+a$};
\node[ordinate, above= of 20] (21) {$y_2=D$};
\node[ordinate, above= of 21] (22) {$y_1=D-a$};
\node[ordinate, above= of 22] (23) {$y_0=D$};

\node[bluedotted, fit =(7)(8)(9)(10)(11)](boxabscissa) {};
\node [above of =7,  node distance=1.2cm] {\textcolor{sthlmBlue}{\textbf{Abscissae}}};

\node[reddotted, fit =(14)(15)(16)(17)(18)](boxordinate1) {};
\node [above of =23,  node distance=1.2cm] {\textcolor{sthlmRed}{\textbf{Ordinates}}};

\node[reddotted, fit =(19)(20)(21)(22)(23)](boxordinate2) {};
\node [above of =18,  node distance=1.2cm] {\textcolor{sthlmRed}{\textbf{Ordinates}}};

\node[pointcircle, left=of 14](25) {$Q_4$};
\node[pointcircle, left=of 15](26) {$Q_3$};
\node[pointcircle, left=of 16](27) {$Q_2$};
\node[pointcircle, left=of 17](28) {$Q_1$};
\node[pointcircle, left=of 18](29) {$Q_0$};

\node[pointcircle, right=of 19](30) {$Q_4$};
\node[pointcircle, right=of 20](31) {$Q_3$};
\node[pointcircle, right=of 21](32) {$Q_2$};
\node[pointcircle, right=of 22](33) {$Q_1$};
\node[pointcircle, right=of 23](34) {$Q_0$};
% Draw Edges
\path [line](1)--(2);
\path [line](2)-- node[color=black]{Yes}(3);
\path [line](3)--(4);
\path [line](4)--(5);
\path [line](2)--node[color=black]{No}(5);
\path [line](5)--(6);
\path [line](6) edge   [bend left=10] (7);
\path [line](7)--(8);
\path [line](8)--(9);
\path [line](9)--(10);
\path [line](10)--(11);
\path [line](12)--(13);
\path [line](11)--(12);
\path [line](13) edge [bend left=30] node[color=black, left]{Yes}(14);
\path [line](13) edge [bend left=-30] node[color=black, right]{No}(19);
\path [line](14)--(15);
\path [line](15)--(16);
\path [line](16)--(17);
\path [line](17)--(18);
\path [line](19)--(20);
\path [line](20)--(21);
\path [line](21)--(22);
\path [line](22)--(23);
\end{tikzpicture}
\end{center}

%-=-=-= EXAMPLE
\begin{example}[id:20150327-115953] \label{20150327-115953}\index{Example!20150327-115953} \hfill \\
Sketch the function $g(x)=\sin(x)+2$

\soln

\begin{sagesilent}
a20150327115953=1
b20150327115953=1
c20150327115953=0
d20150327115953=2

fbase20150327115953=sin(x*pi/180)

f20150327115953=a20150327115953*sin(b20150327115953*(x*pi/180)+c20150327115953*(pi/180))+d20150327115953
ha20150327115953=d20150327115953

phaseshift20150327115953=-c20150327115953/b20150327115953
amplitude20150327115953=abs(a20150327115953)
period20150327115953=360/abs(b20150327115953)
fmaxg20150327115953=d20150327115953+abs(a20150327115953)+0.5
fming20150327115953=d20150327115953-abs(a20150327115953)-0.5
fmax20150327115953=d20150327115953+abs(a20150327115953)
fmin20150327115953=d20150327115953-abs(a20150327115953)

x020150327115953=phaseshift20150327115953
x120150327115953=phaseshift20150327115953+(period20150327115953/4)
x220150327115953=x120150327115953+(period20150327115953/4)
x320150327115953=x220150327115953+(period20150327115953/4)
x420150327115953=x320150327115953+(period20150327115953/4)

x0plot20150327115953=line([(x020150327115953,fmax20150327115953),(x020150327115953,fmin20150327115953)], linestyle="--",zorder=3, rgbcolor=(0.364705882352941, 0.137254901960784, 0.490196078431373), thickness=2)
x1plot20150327115953=line([(x120150327115953,fmax20150327115953),(x120150327115953,fmin20150327115953)], linestyle="--",zorder=3, rgbcolor=(0.364705882352941, 0.137254901960784, 0.490196078431373), thickness=2)
x2plot20150327115953=line([(x220150327115953,fmax20150327115953),(x220150327115953,fmin20150327115953)], linestyle="--",zorder=3, rgbcolor=(0.364705882352941, 0.137254901960784, 0.490196078431373), thickness=2)
x3plot20150327115953=line([(x320150327115953,fmax20150327115953),(x320150327115953,fmin20150327115953)], linestyle="--",zorder=3, rgbcolor=(0.364705882352941, 0.137254901960784, 0.490196078431373), thickness=2)
x4plot20150327115953=line([(x420150327115953,fmax20150327115953),(x420150327115953,fmin20150327115953)], linestyle="--",zorder=3, rgbcolor=(0.364705882352941, 0.137254901960784, 0.490196078431373), thickness=2)

point020150327115953=point((x020150327115953,f20150327115953(x=x020150327115953)), rgbcolor=(0.364705882352941, 0.137254901960784, 0.490196078431373), size=50)
point120150327115953=point((x120150327115953,f20150327115953(x=x120150327115953)), rgbcolor=(0.364705882352941, 0.137254901960784, 0.490196078431373), size=50)
point220150327115953=point((x220150327115953,f20150327115953(x=x220150327115953)), rgbcolor=(0.364705882352941, 0.137254901960784, 0.490196078431373), size=50)
point320150327115953=point((x320150327115953,f20150327115953(x=x320150327115953)), rgbcolor=(0.364705882352941, 0.137254901960784, 0.490196078431373), size=50)
point420150327115953=point((x420150327115953,f20150327115953(x=x420150327115953)), rgbcolor=(0.364705882352941, 0.137254901960784, 0.490196078431373), size=50)

if a20150327115953>0:
    if b20150327115953>0:
        maxmin20150327115953="not reflected"
    else:
        maxmin20150327115953="reflected"
else:
    if b20150327115953>0:
        maxmin20150327115953="reflected"
    else:
        maxmin20150327115953="not reflected"
    
#Define the plots for the trigonometric and sinusoidal axis functions
plotf120150327115953=plot(f20150327115953,(x,x020150327115953,x420150327115953),rgbcolor=(0, 0.431372549019608, 0.749019607843137),thickness=3)
plotf220150327115953=plot(f20150327115953,(x,-180,365),rgbcolor=(0, 0.431372549019608, 0.749019607843137),thickness=3)
plotha20150327115953=plot(ha20150327115953,(x,-180,365),rgbcolor=(0.768627450980392, 0, 0.392156862745098),thickness=3)
plotfbase20150327115953=plot(fbase20150327115953,(x,-180,365), ymin=-1,rgbcolor=(0.2, 0.2, 0.2),thickness=2)

#Define the plots to include in the final plot
plotx20150327115953=plotha20150327115953+x0plot20150327115953+x1plot20150327115953+x2plot20150327115953+x3plot20150327115953+x4plot20150327115953
plotspts20150327115953=plotha20150327115953+x0plot20150327115953+x1plot20150327115953+x2plot20150327115953+x3plot20150327115953+x4plot20150327115953+point020150327115953+point120150327115953+point220150327115953+point320150327115953+point420150327115953
plotfint20150327115953=plotf120150327115953+plotha20150327115953+x0plot20150327115953+x1plot20150327115953+x2plot20150327115953+x3plot20150327115953+x4plot20150327115953+point020150327115953+point120150327115953+point220150327115953+point320150327115953+point420150327115953
plotfcompare20150327115953=plotf220150327115953+plotha20150327115953+plotfbase20150327115953

\end{sagesilent}

\solnsteps

\begin{align*}
	f(x)&=A \sin(Bx+C)+D\\
	g(x)&= \sin(x)+ 2\\
	g(x)&= 1 \sin(1x)+2 && \text{MId} \eqref{eq:mid1} \\
	g(x)&= 1 \sin(1x+0)+2 && \text{AId} \eqref{eq:aid1} 
\end{align*}

Thus, A=1, B=1, C=0 and D=2\\

\textbf{Step 1:}  Find and draw the sinusoidal-axis. \\

\begin{align*}
	y_{sa} &= D \\
	y_{sa} &= 2
\end{align*}


$\sageplot{plotha20150327115953,ymin=fming20150327115953,ymax=fmaxg20150327115953,axes_labels=['$x$','$y$'],ticks=[[-180,-150,-120,-90,-60,-30,0,30, 60,90,120,150,180,210,240,270,300,330,360],1],figsize=[5.5,2.5],tick_formatter="latex",fontsize=7, gridlines="True", frame="True",transparent="True"}$ 

\textbf{Step 2:}  Check if the value of $B<0$ \\

Since $B>0$, there is nothing to do here. \\

\textbf{Step 3:}  Calculate the period $P$

\begin{align*}
	P &= \dfrac{360 \degree}{\abs{B}} \\
	P &= \dfrac{360 \degree}{\abs{\farg{1}}} && \text{SPE} \eqref{eq:spe} \\
	P & = 360 \degree
\end{align*}

\textbf{Step 4:}  Calculate the phase shift $PS$


\begin{align*}
Bx_{ps}+C &=0 \\
1x_{px}+0 &=0 && \text{SPE} \eqref{eq:spe} \\
1x_{px} &= 0 && \text{AId} \eqref{eq:aid2} \\
x_{ps} &=0 && \text{MId} \eqref{eq:mid2} 
\end{align*}

\textbf{Step 5:}  State the abscissae of the some points of interest.

\begin{align*}
x_0 &= PS \\
x_0 &= 0 \degree \\ 
& \\
x_1 &= PS + \dfrac{P}{4} \\
x_1 &= 0 \degree + \dfrac{\farg{360 \degree}}{4} && \text{SPE} \eqref{eq:spe} \\
x_1 &= 0 \degree + 90 \degree \\ %TODO Reduce Fraction
x_1 &= 90 \degree \\ 
& \\
x_2 &= PS + \dfrac{P}{2} \\
x_2 &= 0 \degree + \dfrac{\farg{360 \degree}}{2} && \text{SPE} \eqref{eq:spe} \\
x_2 &= 0 \degree + 180 \degree \\ %TODO Reduce Fraction
x_2 &= 180 \degree \\ 
& \\
x_3 &= PS + \dfrac{3P}{4} \\
x_3 &= 0 \degree + \dfrac{3 \farg{360 \degree}}{4} && \text{SPE} \eqref{eq:spe} \\
x_3 &= 0 \degree + 270 \degree \\ %TODO Reduce Fraction
x_3 &= 270 \degree \\ 
& \\
x_4 &= PS + P \\
x_4 &= 0 \degree + \farg{360 \degree} && \text{SPE} \eqref{eq:spe} \\
x_4 &= 0 \degree + 360 \degree \\ 
x_4 &= 360 \degree
\end{align*}

\textbf{Step 6:}  Sketch the vertical equations of the lines to indicate the $x$ values of the points that we will plot.


$\sageplot{plotx20150327115953,ymin=fming20150327115953,ymax=fmaxg20150327115953,axes_labels=['$x$','$y$'],ticks=[[-180,-150,-120,-90,-60,-30,0,30, 60,90,120,150,180,210,240,270,300,330,360],1],figsize=[5.5,2.5],tick_formatter="latex",fontsize=7, gridlines="True", frame="True",transparent="True"}$ 


\textbf{Step 7:}  Find the amplitude of the function.

\begin{align*}
a & = \abs{A} \\
a & = \abs{\farg{1}} \\
a &= 1 
\end{align*}

\textbf{Step 8:}  Determine if the functions $y$ values are reflected about the $x$-axis.\\

Since the value of $A=1$ therefore $A>0$ and consequently there will be no reflection of the $y$-values about the $x$-axis. \\

\textbf{Step 9:}  Find the corresponding ordinates to the abscissae fond in step 5.

\begin{align*}
y_4 & = D \\
y_4 &= \farg{2} \\
y_4 &= 2 \\
& \\
y_3 & = D-a \\
y_3 &= \farg{2}-\farg{1} \\
y_3 &= 1 \\
& \\
y_2 & = D \\
y_2 &= \farg{2} \\
y_2 &= 2 \\
& \\
y_1 & = D+a \\
y_1 &= \farg{2}+\farg{1} \\
y_1 &= 3 \\
& \\
y_0 & = D \\
y_0 &= \farg{2} \\
y_0 &= 2
\end{align*}

\textbf{Step 10:}  Plot the points of interest.\\

$\sageplot{plotspts20150327115953,ymin=fming20150327115953,ymax=fmaxg20150327115953,axes_labels=['$x$','$y$'],ticks=[[-180,-150,-120,-90,-60,-30,0,30, 60,90,120,150,180,210,240,270,300,330,360],1],figsize=[5.5,2.5],tick_formatter="latex",fontsize=7, gridlines="True", frame="True",transparent="True"}$ 


\textbf{Step 10:}  Sketch the function.\\

$\sageplot{plotfint20150327115953,ymin=fming20150327115953,ymax=fmaxg20150327115953,axes_labels=['$x$','$y$'],ticks=[[-180,-150,-120,-90,-60,-30,0,30, 60,90,120,150,180,210,240,270,300,330,360],1],figsize=[5.5,2.5],tick_formatter="latex",fontsize=7, gridlines="True", frame="True",transparent="True"}$ 

$\sageplot{plotfcompare20150327115953,axes_labels=['$x$','$y$'],ticks=[[-180,-150,-120,-90,-60,-30,0,30, 60,90,120,150,180,210,240,270,300,330,360],1],figsize=[5.5,2.5],tick_formatter="latex",fontsize=7, gridlines="True", frame="True",transparent="True"}$ 

\end{example}



\newpage
\subsubsection{Graphing the Cosine Function}\label{Graphing the Cosine Function}

\begin{center}
\begin{tikzpicture}[scale=1, auto]
% Place nodes
\node[task](1){Draw principal axis $y_{sa}=D$ };
\node[decision, right= of 1] (2) {$B<0$};
\node[task, right= of 2] (3) {$(-Bx\pm C)=-1(Bx \mp C)$};
\node[task, below= of 3] (4) {$A \cos [-1(Bx \mp C)]=A \cos(Bx \mp C)$};
\node[task, left= of 4] (5) {$P= \dfrac{360 \degree}{ \lvert B \rvert}$};
\node[task, left= of 5] (6) {$PS = -\dfrac{C}{B}$};

\node[abscissa, below= of 5, node distance=2.0cm] (7) {$x_0= PS$};
\node[abscissa, below= of 7] (8) {$x_1 = PS + \dfrac{P}{4}$};
\node[abscissa, below= of 8] (9) {$x_2=PS + \dfrac{P}{2}$};
\node[abscissa, below= of 9] (10) {$x_3= PS + \dfrac{3P}{4}$};
\node[abscissa, below= of 10] (11) {$x_4= PS + P$};

\node[task, below= of 11] (12) {$ a=\lvert A \rvert$};
\node[decision, below= of 12] (13) {$A>0$};

\node[ordinate, left= of 11] (14) {$y_4=D+a$};
\node[ordinate, above= of 14] (15) {$y_3=D$};
\node[ordinate, above= of 15] (16) {$y_2=D-a$};
\node[ordinate, above= of 16] (17) {$y_1=D$};
\node[ordinate, above= of 17] (18) {$y_0=D+a$};

\node[ordinate, right= of 11] (19) {$y_4=D-a$};
\node[ordinate, above= of 19] (20) {$y_3=D$};
\node[ordinate, above= of 20] (21) {$y_2=D+a$};
\node[ordinate, above= of 21] (22) {$y_1=D$};
\node[ordinate, above= of 22] (23) {$y_0=D-a$};

\node[bluedotted, fit =(7)(8)(9)(10)(11)](boxabscissa) {};
\node [above of =7,  node distance=1.2cm] {\textcolor{sthlmBlue}{\textbf{Abscissae}}};

\node[reddotted, fit =(14)(15)(16)(17)(18)](boxordinate1) {};
\node [above of =23,  node distance=1.2cm] {\textcolor{sthlmRed}{\textbf{Ordinates}}};

\node[reddotted, fit =(19)(20)(21)(22)(23)](boxordinate2) {};
\node [above of =18,  node distance=1.2cm] {\textcolor{sthlmRed}{\textbf{Ordinates}}};
% Draw Edges
\path [line](1)--(2);
\path [line](2)-- node[color=black]{Yes}(3);
\path [line](3)--(4);
\path [line](4)--(5);
\path [line](2)--node[color=black]{No}(5);
\path [line](5)--(6);
\path [line](6) edge   [bend left=10] (7);
\path [line](7)--(8);
\path [line](8)--(9);
\path [line](9)--(10);
\path [line](10)--(11);
\path [line](12)--(13);
\path [line](11)--(12);
\path [line](13) edge [bend left=30] node[color=black, left]{Yes}(14);
\path [line](13) edge [bend left=-30] node[color=black, right]{No}(19);
\path [line](14)--(15);
\path [line](15)--(16);
\path [line](16)--(17);
\path [line](17)--(18);
\path [line](19)--(20);
\path [line](20)--(21);
\path [line](21)--(22);
\path [line](22)--(23);

\node[pointcircle, left=of 14](25) {$Q_4$};
\node[pointcircle, left=of 15](26) {$Q_3$};
\node[pointcircle, left=of 16](27) {$Q_2$};
\node[pointcircle, left=of 17](28) {$Q_1$};
\node[pointcircle, left=of 18](29) {$Q_0$};

\node[pointcircle, right=of 19](30) {$Q_4$};
\node[pointcircle, right=of 20](31) {$Q_3$};
\node[pointcircle, right=of 21](32) {$Q_2$};
\node[pointcircle, right=of 22](33) {$Q_1$};
\node[pointcircle, right=of 23](34) {$Q_0$};
\end{tikzpicture}
\end{center}


%-=-=-=-=-=-=-=-=-=-=-=-=-=-=-=-=-=-=-=-=-=-=-=-=
%	SECTION: Rational Functions
%-=-=-=-=-=-=-=-=-=-=-=-=-=-=-=-=-=-=-=-=-=-=-=-=

\section{Rational Functions}\label{Rational Functions}

%-=-=-= DEFINITION
\begin{definition}[Asympote]\index{Asympote}

An asymptote is a line or curve that apporoaches a given curve arbitrarily close.\cite{mathworld:asymptote} \\

A vertical asymptote is a vertical line $x_{va}=c$, that is used to visualize the values of $x$ for which the function is not defined. 

\end{definition}

\begin{figure}[ht]
\begin{center}
	\begin{tikzpicture}
	\begin{axis}[
            domain=-6:2,
            ymax=4,
            ymin=-4,
            samples=100,
            axis lines =middle, xlabel=$x$, ylabel=$y$,
            every axis y label/.style={at=(current axis.above origin),anchor=south},
            every axis x label/.style={at=(current axis.right of origin),anchor=west},
			restrict y to domain=-20:20
          ]
          \addplot [dashed, stockholmPink, smooth] plot coordinates {(-2,4) (-2,-4)}; %% {.451};

          \addplot [very thick, stockholmBlue, smooth] {1/(x+2)};

          \node at (axis cs:-5.7,3.8) [anchor=west] {\color{stockholmPink}Vertical Asymptote};  

        \end{axis}
\end{tikzpicture}

\end{center}
\caption{Vertical Asymptote}
\label{figure:rectangularhyperbola}
\end{figure}

%-=-=-= EXAMPLE
\begin{example}[id:20141111-190212] \label{20141111-190212}\index{Example!20141111-190212} \hfill \\

Find the vertical asymptote of the function $R(x)=\dfrac{7}{x+8}$

\soln

\solnsteps

We are interested in the values of $x$ for which the denominator of $R(x)$ has a value of zero.

\begin{align*}
x_{va}+8 &= 0 \\
x_{va} &=-8  &&\text{solving} 
\end{align*}

\begin{center}
	\begin{tikzpicture}
	\begin{axis}[
            domain=-20:10,
            ymax=10,
            ymin=-10,
            samples=100,
            axis lines =middle, xlabel=$x$, ylabel=$y$,
            every axis y label/.style={at=(current axis.above origin),anchor=south},
            every axis x label/.style={at=(current axis.right of origin),anchor=west},
			restrict y to domain=-20:20
          ]
          \addplot [dashed, sthlmRed, smooth] plot coordinates {(-8,10) (-8,-10)}; %% {.451};

          \addplot [very thick, sthlmBlue, smooth] {7/(x+8)};

          \node at (axis cs:-17.7,5.8) [anchor=west] {\color{sthlmRed}$x_{va}=-8$};  

        \end{axis}
\end{tikzpicture}

\end{center}
\end{example}

%-=-=-= EXAMPLE
\begin{example}[id:20141111-192213] \label{20141111-192213}\index{Example!20141111-192213} \hfill \\

Find the vertical asymptote of the function $R(x)=\dfrac{x+4}{2x+5}$

\soln

\solnsteps

We are interested in the values of $x$ for which the denominator of $R(x)$ has a value of zero.

\begin{align*}
2x_{va}+5 &= 0 \\
x_{va} &= -\dfrac{5}{2}  &&\text{solving \ref{20141111-215726}} \\
\end{align*}

\begin{center}
	\begin{tikzpicture}
	\begin{axis}[
            domain=-6:2,
            ymax=5,
            ymin=-5,
            samples=100,
            axis lines =middle, xlabel=$x$, ylabel=$y$,
            every axis y label/.style={at=(current axis.above origin),anchor=south},
            every axis x label/.style={at=(current axis.right of origin),anchor=west},
			restrict y to domain=-5:5
          ]
          \addplot [dashed, sthlmRed, smooth] plot coordinates {(-5/2,4) (-5/2,-4)}; %% {.451};

          \addplot [very thick, sthlmBlue, smooth] {(x+4)/(2*x+5)};

          \node at (axis cs:-4.7,2.8) [anchor=west] {\color{sthlmRed}$x_{va}=-\frac{5}{2}$};  

        \end{axis}
\end{tikzpicture}

\end{center}
\end{example}







