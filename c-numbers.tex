\documentclass[20150903-160354-rs2.2-MarksMathNotebook.tex]{subfiles}

\begin{document}
%-=-=-=-=-=-=-=-=-=-=-=-=-=-=-=-=-=-=-=-=-=-=-=-=
%
%	CHAPTER
%
%-=-=-=-=-=-=-=-=-=-=-=-=-=-=-=-=-=-=-=-=-=-=-=-=

\chapter{Numbers}


%-=-=-=-=-=-=-=-=-=-=-=-=-=-=-=-=-=-=-=-=-=-=-=-=
%	SECTION:
%-=-=-=-=-=-=-=-=-=-=-=-=-=-=-=-=-=-=-=-=-=-=-=-=

\section{Number Systems}\index{Number Systems}

%-=-=-= DEFINITION
\begin{definition}[Natural Numbers]\index{Number System! Natural Numbers}

\[
\mathbb{N}=\set{0, 1, 2, 3 \ldots}
\]

\end{definition}

\begin{remark}
It is not uncommon for zero to be excluded from the natural numbers.  In fact, some exclude zero from the natural numbers and then describe the set of natural numbers that include zero the whole numbers. \\

\[
\mathbb{W}=\set{0, 1, 2, 3, \ldots}
\]

For the purposes of these notes, zero will be included within the set of natural numbers.
\end{remark}

%-=-=-= DEFINITION
\begin{definition}[Integers]\index{Number System! Integers}

\[
\mathbb{Z}=\set{\ldots, -3, -2, -1, 0, 1, 2, 3, \ldots}
\]

\end{definition}

%-=-=-= DEFINITION
\begin{definition}[Positive Integers]\index{Positive Integers}

\[
\mathbb{Z}^{+}=\set{1, 2, 3, \ldots}
\]
\end{definition}

%-=-=-= DEFINITION
\begin{definition}[Rational Numbers]\index{Number System! Rational Numbers}

\[
\mathbb{Q}=\set{m/n \suchthat m,n \in \mathbb{Z}, n \ne 0}
\]
\end{definition}

%-=-=-= DEFINITION
\begin{definition}[Proper Fraction]\index{Proper Fraction}

Given $m<n$, then the fraction $m/n$ is called \alert{proper}.

\end{definition}

%-=-=-= DEFINITION
\begin{definition}[Improper Faction]\index{Improper Faction}

Given $m>n$, then the fraction $m/n$ is called \alert{improper}.

\end{definition}

%-=-=-=-=-=-=-=-=-=-=-=-=-=-=-=-=-=-=-=-=-=-=-=-=
%	SECTION:
%-=-=-=-=-=-=-=-=-=-=-=-=-=-=-=-=-=-=-=-=-=-=-=-=

\section{Prime Numbers}\index{Prime Numbers}

%-=-=-= DEFINITION
\begin{definition}[Greatest Common Divisor]\index{Greatest Common Divisor}

Suppose that $m$ and $n$ are positive integers.  The greatest common divisor is the largest divisor (factor) common to both $m$ and $n$.

\end{definition}

%-=-=-= DEFINITION
\begin{definition}[Relatively Prime]\index{Relatively Prime}

Two integers $m$ and $n$ are relatively prime to each other, $m \perp n$, if they share no common positive integer divisors (factors) except 1.

\[
m \perp n \, \text{if} \, \gcd(m, n)=1.
\]
\end{definition}
\subsection{Listing of Prime Numbers 2-997}
\halign{
#	& 	# 	&	#	&	#	&	#	&	#	&	#	&	#	&	#	&	#	&	#	&	#	\cr
	& 		& 		& 		&  		&  		&  		&  		&  		&  		& 		&		\cr
2 	& 	3 	& 	5 	& 	7 	& 	11 	& 	13 	& 	17 	& 	19 	& 	23 	& 	29 	& 	31 	& 	37 	\cr
41 	&	43 	& 	47 	&	53 	& 	59 	& 	61 	& 	67 	& 	71 	& 	73 	& 	79 	& 	83 	& 	89 	\cr
97 	& 	101 & 	103 & 	107 & 	109 &	113 & 	127 & 	131 & 	137 & 	139 & 	149 & 	151 \cr
157 & 	163 &	167 &	173 & 	179 &	181 & 	191 & 	193 & 	197 & 	199 & 	211 & 	223 \cr
227 & 	229 & 	233 & 	239 &	241 & 	251 & 	257 & 	263 & 	269 & 	271 & 	277 & 	281 \cr
283 & 	293 & 	307 & 	311 & 	313 &	317 & 	331 & 	337 & 	347 & 	349 & 	353 & 	359 \cr
367 & 	373 & 	379 & 	383 & 	389 & 	397 &	401 & 	409 & 	419 & 	421 & 	431 & 	433 \cr
439 & 	443 & 	449 & 	457 & 	461 &	463 & 	467 &	479 & 	487 & 	491 & 	499 & 	503 \cr
509 & 	521 & 	523 & 	541 & 	547 &	557 & 	563 & 	569 &	571 & 	577 & 	587 & 	593 \cr
599 & 	601 & 	607 & 	613 & 	617 &	619 & 	631 & 	641 & 	643 &	647 & 	653 & 	659 \cr
661 & 	673 & 	677 & 	683 & 	691 &	701 & 	709 & 	719 & 	727 & 	733 &	739 & 	743 \cr
751 & 	757 & 	761 & 	769 & 	773 &	787 & 	797 & 	809 & 	811 & 	821 & 	823 &	827 \cr
829 & 	839 & 	853 & 	857 & 	859 &	863 & 	877 & 	881 & 	883 &	887 & 	907 & 	911 \cr
919 & 	929 & 	937 & 	941 & 	947	&	953 & 	967 & 	971 & 	977 & 	983 & 	991 & 	997 \cr
}


\end{document}

