\documentclass[20150903-160354-rs2.2-MarksMathNotebook.tex]{subfiles}

\begin{document}
%-=-=-=-=-=-=-=-=-=-=-=-=-=-=-=-=-=-=-=-=-=-=-=-=
%
%	CHAPTER
%
%-=-=-=-=-=-=-=-=-=-=-=-=-=-=-=-=-=-=-=-=-=-=-=-=

\chapterimage{Pictures/chapter_head_2.pdf} % Chapter heading image

\chapter{Algebraic Expressions}

%-=-=-=-=-=-=-=-=-=-=-=-=-=-=-=-=-=-=-=-=-=-=-=-=
%	SECTION: OPERATION OF ADDITION
%-=-=-=-=-=-=-=-=-=-=-=-=-=-=-=-=-=-=-=-=-=-=-=-=
\section{Expressions}\index{Algebraic Expressions}

\begin{essentialq}\hfill \\

\begin{enumerate}
	\item What is an algebraic expression?
\end{enumerate}

\end{essentialq}


\begin{table}
\begin{tabular}{|l|c|c|c|c|c|c}
\hline
 									& Arithmetic 	& Polynomial	& Algebraic  	\\
\hline
Constant 							& \cellyes		& \cellyes 		& \cellyes		\\
\hline
Factorial 							& \cellyes 		& \cellyes 		& \cellyes		\\
\hline
Variable: parameter/coefficient 	& \cellyes 		& \cellyes 		& \cellyes		\\
\hline
Variable: unknown/indeterminate 	& \cellno 		& \cellyes 		& \cellyes		\\
\hline
Power with $\mathbb{Z}^{+}$ exponent 	& \cellno 		& \cellyes 		& \cellyes		 \\
\hline
Power with $\mathbb{Z}$ exponent 		& \cellno 		& \cellno 		& \cellyes		 \\
\hline
$n$-th root 						& \cellno 		& \cellno 		& \cellyes		\\
\hline
Power with $\mathbb{Q}$ exponent 	& \cellno 		& \cellno 		& \cellyes		 \\
\hline
\end{tabular}
\caption{Names of different types of expressions}\label{tab:tableofexpressions}
\end{table}


%-=-=-= DEFINITION
%\begin{definition}[Algebraic Expression]\index{Algebraic Expression}

%An algebraic expression is made up of one or more terms (operation of addition), which are themselves made up of factors (operation of multiplication).\\

%Algebraic expressions have three different types of multiplicative factors:\\

%\begin{enumerate}
%	\item constant (coefficient)
%	\begin{itemize}
%		\item real number
%		\item parameter
%	\end{itemize}
%	\item variable
%	\item rational power
%\end{enumerate}
%\hfill \cite{mathworld:algebraicexpression}
%\end{definition}


\subsection{Surds}\index{Surds}

%-=-=-= EXAMPLE
\begin{example}[id:20141108-085327] \label{20141108-085327} \index{Example!20141108-085327} \hfill \\

Simplify $2\sqrt{2}-\dfrac{\left(\sqrt{2}\right)^3}{3}-\left(2\left(-\sqrt{2}\right)-\dfrac{\left(-\sqrt{2}\right)^3}{3} \right)$

\soln

\solnsteps
\begin{align*}
2 \sqrt{2}- \dfrac{\alert{1}\left(\alert{1}\sqrt{2}\right)^{3}}{3}-\alert{1} \left(2 \left(-\alert{1}\sqrt{2} \right) - \dfrac{\alert{1} \left(-\alert{1} \sqrt{2}\right)^{3}}{3} \right) && \text{MId} \eqref{eq:mid1} \\
2 \sqrt{2} - \dfrac{1 \left(1\sqrt{2}\right)^{3}}{3}-1 \left(2 \left(\alert{\neg 1} \sqrt{2} \right) - \dfrac{1 \left(\alert{\neg 1} \sqrt{2}\right)^{3}}{3} \right) && \text{ONeg} \eqref{eq:oneg1} \\
2 \sqrt{2} + \dfrac{\neg 1 \left(1\sqrt{2}\right)^{3}}{3}+ \neg 1 \left(2 \left(\alert{\neg 1} \sqrt{2} \right) + \dfrac{\neg 1 \left(\alert{\neg 1} \sqrt{2}\right)^{3}}{3} \right) && \text{DOS} \eqref{eq:dos1} \\
2 \cdot \alert{2^{1/2}} + \dfrac{\neg 1 \left(1 \cdot \alert{2^{1/2}}\right)^{3}}{3} + \neg 1 \left(2 \left(\neg  1 \cdot \alert{2^{1/2}} \right) + \dfrac{\neg 1 \left(\neg  1  \cdot \alert{2^{1/2}}\right)^{3}}{3} \right) && \text{RTPo} \eqref{eq:rtpo} \\
2 \cdot 2^{1/2} + \dfrac{\neg 1 \left(1 \cdot 2^{1/2}\right)^{3}}{3} + \neg 1 \left(\alert{2 \cdot \neg  1 \cdot 2^{1/2}} + \dfrac{\neg 1 \left(\neg 1  \cdot 2^{1/2}\right)^{3}}{3} \right) && \text{JTC} \eqref{eq:jtc} \\
2 \cdot 2^{1/2} + \dfrac{\neg 1 \cdot \alert{1 \cdot 2^{3/2}}}{3} + \neg 1 \left(2 \cdot \neg  1 \cdot 2^{1/2} + \dfrac{\neg 1 \cdot \alert{\neg 1  \cdot 2^{3/2}}}{3} \right) && \text{PoPrPo} \eqref{eq:poprpo1} \\
2 \cdot 2^{1/2} + \dfrac{\neg 1 \cdot 1 \cdot \alert{2^{2/2} \cdot 2^{1/2}}}{3} + \neg 1 \left(2 \cdot \neg  1 \cdot 2^{1/2} + \dfrac{\neg 1 \cdot \neg 1  \cdot \alert{2^{2/2} \cdot 2^{1/2}}}{3} \right) && \text{PrCBPo} \eqref{eq:prcbpo2} \\
2 \cdot 2^{1/2} + \dfrac{\neg 1 \cdot 1 \cdot \alert{2} \cdot 2^{1/2}}{3} + \neg 1 \left(2 \cdot \neg  1 \cdot 2^{1/2} + \dfrac{\neg 1 \cdot \neg 1  \cdot \alert{2} \cdot 2^{1/2}}{3} \right) && \text{MId} \eqref{eq:mid2} \\
2 \cdot \alert{\sqrt{2}} + \dfrac{\neg 1 \cdot 1 \cdot 2 \cdot \alert{\sqrt{2}}}{3} + \neg 1 \left(2 \cdot \neg  1 \cdot \alert{\sqrt{2}} + \dfrac{\neg 1 \cdot \neg 1  \cdot 2 \cdot \alert{\sqrt{2}}}{3} \right) && \text{PoTR} \eqref{eq:potr} \\
2 \cdot \sqrt{2} + \dfrac{\neg 1 \cdot 1 \cdot 2 \cdot \sqrt{2}}{3} + \alert{\neg 1} \cdot 2 \cdot \neg  1 \cdot \sqrt{2} + \dfrac{\alert{\neg 1} \cdot \neg 1 \cdot \neg 1  \cdot 2 \cdot \sqrt{2}}{3}  && \text{DPE} \eqref{eq:dpe1} \\
2 \cdot \sqrt{2} + \dfrac{\neg 2 \cdot \sqrt{2}}{3} + 2 \cdot \sqrt{2} + \dfrac{\neg 2 \cdot \sqrt{2}}{3}  && \text{OOM} \eqref{eq:oom} \\
2\sqrt{2} + \dfrac{\neg 2\sqrt{2}}{3} + 2\sqrt{2} + \dfrac{\neg 2\sqrt{2}}{3} && \text{CTJ} \eqref{eq:ctj} \\
\left(2 + \dfrac{\neg 2}{3} + 2 + \dfrac{\neg 2}{3} \right)  \sqrt{2} && \text{DPF} \eqref{eq:dpf1} \\
\dfrac{8}{3}\sqrt{2} && \text{OOA} \eqref{eq:ooa}
\end{align*}

\qdepend

\qdependlist
example \ref{20141108-083108}-20141108-083108

\end{example}

\end{document}

