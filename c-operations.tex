\documentclass[20150903-160354-rs2.2-MarksMathNotebook.tex]{subfiles}

\begin{document}
%-=-=-=-=-=-=-=-=-=-=-=-=-=-=-=-=-=-=-=-=-=-=-=-=
%
%	CHAPTER
%
%-=-=-=-=-=-=-=-=-=-=-=-=-=-=-=-=-=-=-=-=-=-=-=-=

\chapter{Operations}


%-=-=-=-=-=-=-=-=-=-=-=-=-=-=-=-=-=-=-=-=-=-=-=-=
%	SECTION:
%-=-=-=-=-=-=-=-=-=-=-=-=-=-=-=-=-=-=-=-=-=-=-=-=

\section{Dyadic Operations}\index{Dyadic Operations}


%-=-=-= DEFINITION
\begin{definition}[Operation of Addition (OOA)]\index{Operation!Operation of Addition}
\begin{align}
\underbrace{\underbrace{a}_{\text{Augend}}+\underbrace{b}_{\text{Addend}}}_{\text{Sum}} \label{eq:ooa}
\end{align}
More generally,
\begin{align}
\underbrace{\underbrace{a}_{\text{Summand}}+\underbrace{b}_{\text{Summand}}}_{\text{Sum}} \label{eq:ooag}
\end{align}
\end{definition}

%-=-=-= DEFINITION
\begin{definition}[Operation of Subtraction (OOS)]\index{Operation!Operation of Subtraction}
\begin{align}
\underbrace{\underbrace{a}_{\text{Minuend}}-\underbrace{b}_{\text{Subtrahend}}}_{\text{Difference}} \label{eq:oos}
\end{align}
\end{definition}

%-=-=-= DEFINITION
\begin{definition}[Operation of Multiplication (OOM)] \index{Operation!Operation of Multiplication}
\begin{align}
\underbrace{\underbrace{a}_{\text{Multiplicand}} \times \underbrace{b}_{\text{Multiplier}}}_{\text{Product}} \label{eq:oom}
\end{align}
More generally,
\begin{align}
\underbrace{\underbrace{a}_{\text{Factor}} \times \underbrace{b}_{\text{Factor}}}_{\text{Product}} \label{eq:oomg}
\end{align}
\end{definition}

%-=-=-= DEFINITION
\begin{definition}[Operation of Exponentiation (OOE)]\index{Operation!Operation of Exponentiation}
\begin{align}
\underbrace{{\underbrace{b}_{base}}^{\overbrace{m}^{Exponent}}}_{Power} \label{eq:ooe}
\end{align}
\end{definition}

%-=-=-= DEFINITION
\begin{definition}[Common Denominator (CD)]\index{Common Denominator}
\begin{subequations}
\begin{align}
\dfrac{a}{b} + \dfrac{c}{b} &= \dfrac{a+c}{b} \label{eq:cd1} \\
\dfrac{a+c}{b}&= \dfrac{a}{b} + \dfrac{c}{b} \label{eq:cd2}
\end{align}
\end{subequations}
\end{definition}

%-=-=-= RULE
\begin{arule}[Fraction Operation of Addition (FOOA)]\index{Fraction Operation of Addition}
\begin{subequations}
\begin{align}
\dfrac{a}{b} + \dfrac{c}{d} &= \dfrac{ad+bc}{bd} \label{eq:fooa1} \\
\dfrac{ad+bc}{bd} &= \dfrac{a}{b} + \dfrac{c}{d} \label{eq:fooa2}
\end{align}
\end{subequations}
\end{arule}

\end{document}

