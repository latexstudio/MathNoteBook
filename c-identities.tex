\documentclass[20150903-160354-rs2.2-MarksMathNotebook.tex]{subfiles}

\begin{document}
%-=-=-=-=-=-=-=-=-=-=-=-=-=-=-=-=-=-=-=-=-=-=-=-=
%
%	CHAPTER
%
%-=-=-=-=-=-=-=-=-=-=-=-=-=-=-=-=-=-=-=-=-=-=-=-=

\chapter{Identities}

%-=-=-=-=-=-=-=-=-=-=-=-=-=-=-=-=-=-=-=-=-=-=-=-=
%	SECTION:
%-=-=-=-=-=-=-=-=-=-=-=-=-=-=-=-=-=-=-=-=-=-=-=-=

\section{Power Identities}\index{Identities! Power Identities}

\begin{aidentity}[Power of Power (PoPo)]\index{Powers!Power of a Power}
\begin{subequations}
\begin{align}
	(b^m)^k &=b^{m \cdot k} \label{eq:popo1}\\
	b^{m \cdot k} &= (b^m)^k \label{eq:popo2}
\end{align}
\end{subequations}
\end{aidentity}

\begin{aidentity}[Power of a Product (PoPr)]\index{Powers!Power of a Power}
\begin{subequations}
\begin{align}
	(a \cdot b)^k &=a^k \cdot b^k \label{eq:popr1}\\
	a^k \cdot b^k &= (a \cdot b)^k \label{eq:popr2}
\end{align}
\end{subequations}
\end{aidentity}

\begin{aidentity}[Product Common Base Powers (PrCBPo)]\index{Powers!Product Common Base Powers}
\begin{subequations}
\begin{align}
	b^m \cdot b^n &=b^{m+n} \label{eq:prcbpo1}\\
	b^{m+n} &= b^m \cdot b^n \label{eq:prcbpo2}
\end{align}
\end{subequations}
\end{aidentity}

\begin{aidentity}[Quotient Common Base Powers (QCBPo)]\index{Powers!Quotient Common Base Powers}
\begin{subequations}
\begin{align}
	\frac{b^m}{b^n} &= b^{m-n} \label{eq:qcbpo1}\\
	b^{m-n} &= \frac{b^m}{b^n} \label{eq:qcbpo2}
\end{align}
\end{subequations}
\end{aidentity}

\begin{aidentity}[Power of a Quotient of Powers (PoQPo)]\index{Powers!Power of a Quotient of Powers}
\begin{subequations}
\begin{align}
	\left(\dfrac{a^m}{b^n}\right)^k &= \dfrac{a^{m \cdot k}}{b^{n\cdot k}} \label{eq:poqpo1}\\
	\dfrac{a^{m \cdot k}}{b^{n\cdot k}} &= \left(\dfrac{a^m}{b^n}\right)^k \label{eq:poqpo2}
\end{align}
\end{subequations}
\end{aidentity}

\begin{aidentity}[Power of a Product of Powers (PoPrPo)]\index{Powers!Power of a Product of Powers}
\begin{subequations}
\begin{align}
	\left(a^m \cdot b^n\right)^k &= a^{m \cdot k} \cdot b^{n\cdot k} \label{eq:poprpo1}\\
	a^{m \cdot k} \cdot b^{n \cdot k} &= \left(a^m \cdot b^n\right)^k \label{eq:poprpo2}
\end{align}
\end{subequations}
\end{aidentity}

%-=-=-=-=-=-=-=-=-=-=-=-=-=-=-=-=-=-=-=-=-=-=-=-=
%	SECTION:
%-=-=-=-=-=-=-=-=-=-=-=-=-=-=-=-=-=-=-=-=-=-=-=-=

\section{Logarithm Identities}\index{Identities! Power Identities}

\begin{aidentity}[Logarithm Power of a Power (LPoPo)]\index{LogarithmIdentity!Logarithm Power of a Power}
\begin{subequations}
\begin{align}
	\log_b x^n &= n \log_b x \label{eq:lpopo1}\\
	n \log_b x &= \log_b x^n \label{eq:lpopo2}
\end{align}
\end{subequations}
\end{aidentity}

\begin{aidentity}[Logarithm Product of Common Base Powers (LPrCBPo)]\index{LogarithmIdentity!Logarithm Product of Common Base Powers}
\begin{subequations}
\begin{align}
	\log_b (mn) &= \log_b m + \log_b n \label{eq:lprcbpo1}\\
	\log_b m + \log_b n &= \log_b (mn) \label{eq:lprcbpo2}
\end{align}
\end{subequations}
\end{aidentity}

\begin{aidentity}[Logarithm Quotient of Common Base Powers (LQCBPo)]\index{LogarithmIdentity!Logarithm Quotient of Common Base Powers}
\begin{subequations}
\begin{align}
	\log_b \left( \frac{m}{n} \right) &= \log_b m - \log_b n \label{eq:lqcbpo1}\\
	 \log_b m - \log_b n &= \log_b \left( \frac{m}{n} \right) \label{eq:lqcbpo2}
\end{align}
\end{subequations}
\end{aidentity}

%-=-=-=-=-=-=-=-=-=-=-=-=-=-=-=-=-=-=-=-=-=-=-=-=
%	SECTION:
%-=-=-=-=-=-=-=-=-=-=-=-=-=-=-=-=-=-=-=-=-=-=-=-=

\section{Trigonometric Identities}\index{Identities! Power Identities}
\end{document}

\begin{aidentity}[Trigonometric Reciprocal Identities (TRId)]\index{TrigonometricIdentity!Trigonometric Reciprocal Identities}
\begin{subequations}
\begin{align}
	\sin \theta &= \frac{1}{\csc \theta} \label{eq:sinrid}\\
	\cos \theta &= \frac{1}{\sec \theta} \label{eq:cosrid}\\
	\cot \theta &= \frac{1}{\tan \theta} \label{eq:cotrid}\\
	\csc \theta &= \frac{1}{\sin \theta} \label{eq:cscrid}\\
	\sec \theta &= \frac{1}{\cos \theta} \label{eq:secrid}\\
	\tan \theta &= \frac{1}{\cot \theta} \label{eq:tanrid}
\end{align}
\end{subequations}
\end{aidentity}

\begin{aidentity}[Trigonometric Pythagorean Identities (TPythagId)]\index{TrigonometricIdentity!Trigonometric Pythagorean Identities}
\begin{subequations}
\begin{align}
	\sin^2 \theta + \cos^2 \theta & = 1 \label{eq:tpythagid1}\\
	\sec^2 \theta & = \tan^2 \theta +1 \label{eq:tpythagid2}\\
	\csc^2 \theta & = 1 + \cot^2 \theta \label{eq:tpythagid3}
\end{align}
\end{subequations}
\end{aidentity}

\begin{aidentity}[Trigonometric Tangent Identity (TanId)]\index{TrigonometricIdentity!Trigonometric Tangent Identity}
\begin{subequations}
\begin{align}
	\tan \theta &= \frac{\sin \theta}{\cos \theta} \label{eq:tanid}
\end{align}
\end{subequations}
\end{aidentity}

\begin{aidentity}[Trigonometric Cotangent Identity (CotId)]\index{TrigonometricIdentity!Trigonometric Cotangent Identity}
\begin{subequations}
\begin{align}
	\cot \theta &= \frac{\cos \theta}{\sin \theta} \label{eq:cotid}
\end{align}
\end{subequations}
\end{aidentity}

\begin{aidentity}[Sine Double Angle Identity (SinDAId)]\index{TrigonometricIdentity!Sine Double Angle Identity}
\begin{subequations}
\begin{align}
	\sin 2\theta &= 2 \sin \theta \cos \theta \label{eq:sindaid}
\end{align}
\end{subequations}
\end{aidentity}

\begin{aidentity}[Cosine Double Angle Identity (CosDAId)]\index{TrigonometricIdentity!Cosine Double Angle Identity}
\begin{subequations}
\begin{align}
	\cos 2\theta & = \cos^2 \theta - \sin^2 \theta \label{eq:cosdaid1}\\
	\phantom{\cos 2\theta} &=  1-2 \sin^2 \theta \label{eq:cosdaid2}\\
	\phantom{\cos 2\theta} &=  2 \cos^2 \theta-1 \label{eq:cosdaid3}
\end{align}
\end{subequations}
\end{aidentity}

\begin{aidentity}[Tangent Double Angle Identity (TanDAId)]\index{TrigonometricIdentity!Tangent Double Angle Identity}
\begin{subequations}
\begin{align}
	\tan 2\theta &= \frac{2 \tan \theta}{1-\tan^2 \theta} \label{eq:tandaid}
\end{align}
\end{subequations}
\end{aidentity}

\begin{aidentity}[Sine Sum of Angles Identity (SinSAId)]\index{TrigonometricIdentity!Sine Sum of Angles Identity}
\begin{subequations}
\begin{align}
	\sin (\theta + \phi) &= \sin \theta \cos \phi + \cos \theta \sin \phi \label{eq:sinsaid}
\end{align}
\end{subequations}
\end{aidentity}

\begin{aidentity}[Sine Difference of Angles Identity (SinDiffAId)]\index{TrigonometricIdentity!Sine Difference of Angles Identity}
\begin{subequations}
\begin{align}
	\sin (\theta - \phi) &= \sin \theta \cos \phi - \cos \theta \sin \phi \label{eq:sindiffaid}
\end{align}
\end{subequations}
\end{aidentity}

\begin{aidentity}[Cosine Sum of Angles Identity (CosSAId)]\index{TrigonometricIdentity!Cosine Sum of Angles Identity}
\begin{subequations}
\begin{align}
	\cos (\theta + \phi) &= \cos \theta \cos \phi - \sin \theta \sin \phi \label{eq:cossaid}
\end{align}
\end{subequations}
\end{aidentity}

\begin{aidentity}[Cosine Difference of Angles Identity (CosDiffAId)]\index{TrigonometricIdentity!Cosine Difference of Angles Identity}
\begin{subequations}
\begin{align}
	\cos (\theta - \phi) &= \cos \theta \cos \phi + \sin \theta \sin \phi \label{eq:cosdiffaid}
\end{align}
\end{subequations}
\end{aidentity}

\begin{aidentity}[Tangent Sum of Angles Identity (TanSAId)]\index{TrigonometricIdentity!Tangent Sum of Angles Identity}
\begin{subequations}
\begin{align}
	\tan (\theta + \phi) &= \dfrac{\tan \theta + \tan \phi}{1- \tan \theta \tan \phi} \label{eq:tansaid}
\end{align}
\end{subequations}
\end{aidentity}

\begin{aidentity}[Tangent Difference of Angles Identity (TanDiffAId)]\index{TrigonometricIdentity!Tangent Difference of Angles Identity}
\begin{subequations}
\begin{align}
	\tan (\theta - \phi) & = \dfrac{\tan \theta - \tan \phi}{1+ \tan \theta \tan \phi} \label{eq:tandiffaid}
\end{align}
\end{subequations}
\end{aidentity}



